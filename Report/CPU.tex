\documentclass[]{report}
\usepackage[left=0.8in, right=0.8in, top=1in, bottom=1in]{geometry}
\usepackage{listings}
\usepackage{xcolor}
\usepackage{amsmath}
\usepackage{amsthm}
\usepackage{amssymb}
\usepackage{esint}
\usepackage[parfill]{parskip}
\usepackage{hyperref}
\usepackage{floatrow}
\usepackage{mwe}
\usepackage{graphicx}
\usepackage{multirow}
\usepackage{multicol}
\usepackage{enumitem}
\usepackage{setspace}
\usepackage{biblatex}
\usepackage[none]{hyphenat}

\title{EE 224 Course Project : CPU}

\author{Atharva Kulkarni(210070047) \\Harshit Raj(20d070033) \\
 Shreyas Grampurohit(21d070029) \\ Varad Deshpande(21d070024)}

\date{9 Nov 2022}

\begin{document}

\maketitle
\section*{ALU:}
\subsection*{Data-flow}
\begin{figure}[H]
    \begin{center}
        \includegraphics*[width=0.85\textwidth]{ADDNAND.jpg}
    \end{center}
\end{figure}
    \begin{figure}[H]
        \begin{center}
            \includegraphics*[width=0.85\textwidth]{BEQ+JAL+JLR.jpg}
        \end{center}
    \end{figure}
We have separated data-flow diagrams for ADD, NAND instructions from other instructions for simplicity.
    \subsection*{Inputs:}
    ALU\_A: Takes 3-bit input \\
    ALU\_B: Takes 3-bit input \\

    \subsection*{Registers storing the flags:}
    FC and FZ \\
    These are connected to ALU\_C and ALU\_Z respectively.

    \subsection*{Control Signals:}
    ALU\_J: Takes 2-bit input. This specifies whether to perform addition, subtraction or NAND. \\
    ALU\_CND: Takes 2-bit input. This is used to find the new values to be updated to FZ and FC(may or may not be the same as the previous values in FZ and FC) based upon ALU\_J(see table). \\

    \subsection*{Outputs:}
    ALU\_C: Outputs the carry flag to be put in the FZ register at the end of the clock cycle. \\
    ALU\_Z: Outputs the zero flag to be put in the FZ register at the end of the clock cycle. This may or may not be the same as Z\_int. \\
    ALU\_S: Outputs sum, NAND, or the difference based on the bits provided in ALU\_J. \\
    Z\_int: Evaluates to 1 when ALU\_S is zero. Else, it evaluates to 0. \\

\begin{center}
    \begin{tabular}{|c|c|}
        \hline
        ALU\_J & Function \\
        \hline
        00 & Addition \\
        \hline
        01 & NAND \\
        \hline
        11 & Subtraction \\
        \hline
    \end{tabular}
\end{center}

\begin{table}[ht]
\begin{center}
\begin{tabular}{|p{0.13\linewidth}|p{0.16\linewidth}|p{0.3\linewidth}|p{0.3\linewidth}|}
    \hline
    ALU\_J & ALU\_CND & Output from ALU\_C & Output from ALU\_Z  \\
    \hline
    00 (Add) & 00 & Modified value of carry flag. & Modified value of zero flag. \\
    \hline
    00 (Add) & 10 & Modified value of carry flag if input FC is 1. Same as the previous value in FC if FC is 0. & Modified value of zero flag if input FC is 1. Same as the previous value in FZ if FC is 0. \\
    \hline
    00 (Add) & 01 & Modified value of carry flag if input FZ is 1. Same as the previous value in FC if FZ is 0. & Modified value of zero flag if input FZ is 1. Same as the previous value in FZ if FZ is 0 \\
    \hline
    00 (Add) & 11 (Used for updating PC) & Same as the previous value in FC. & Same as the previous value in FZ. \\
    \hline
    01 (NAND) & 00 & Same as the previous value in FC. & Modified value of zero flag. \\
    \hline
    01 (NAND) & 10 & Same as the previous value in FC. & Modified value of zero flag if input FC is 1. Same as the previous value in FZ if FC is 0. \\
    \hline
    01 (NAND) & 01 & Same as the previous value in FC. & Modified value of zero flag if input FZ is 1. Same as the previous value in FZ if FZ is 0. \\
    \hline
    11 (Subtract) & xx & Same as the previous value in FC. & Same as the previous value in FZ. \\
    \hline
\end{tabular}
\end{center}
\end{table}

\newpage
% remember to mention that PC eqt to R7
    \section*{State Descriptions}
        (PC $\equiv$ R7) \\ 
        \subsection*{$S_0$ (Fetching instruction from memory)}  % can be merged with S1
            \begin{center}
                \begin{tabular}{|c|c|}
                    \hline
                    Data Transfer & Commands \\
                    \hline
                    PC $\to$ M\_add & MDR\\
                        M\_data $\to$ T1& T1\_E\\
                    \hline
                \end{tabular}
            \end{center}
        \subsection*{$S_1$ (Updating PC)}   % +1 or +2
        \begin{center}
            \begin{tabular}{|c|c|}
                \hline
                Data Transfer & Commands \\
                \hline
                PC $\to$ ALU\_A & PC\_E\\
                +1 $\to$ ALU\_B & ALU\_J $\leftarrow$ 00\\
                ALU\_CND $\leftarrow$ 11 & \\
                ALU\_S $\to$ PC &\\
                \hline
            \end{tabular}
        \end{center}
        \subsection*{$S_2$ (Reading operands)}
        \begin{center}
            \begin{tabular}{|c|c|}
                \hline
                Data Transfer & Commands \\
                \hline
                $T1_{11-9} \to$ RF\_A1  & T2\_E\\
                $T1_{8-6} \to$ RF\_A2 & T3\_E\\
                RF\_D1 $\to$ T2 & \\
                RF\_D2 $\to$ T3 & \\
                \hline
            \end{tabular}
        \end{center}
        \subsection*{$S_3$ (Execution)} % can execute ADD, SUB, NAND
        \begin{center}
            \begin{tabular}{|c|c|}
                \hline
                Data Transfer & Commands \\
                \hline
                T2 $\to$ ALU\_A & T2\_E\\
                T3 $\to$ ALU\_B & ALU\_J $\leftarrow$ $T1_{14-13}$\\
                $T1_{1-0} \to $ ALU\_CND & \\
                ALU\_S $\to$ T2 & \\
                ALU\_C $\to$ FC & \\
                ALU\_Z $\to$ FZ & \\
                \hline
            \end{tabular}
        \end{center}
        \subsection*{$S_4$ (Storing the output)}
        \begin{center}
            \begin{tabular}{|c|c|}
                \hline
                Data Transfer & Commands \\
                \hline
                T2 $\to$ RF\_D3 & RF\_WR\\
                $T1_{8-6}$ $\to$ RF\_A3 & \\                
                \hline
            \end{tabular}
        \end{center}
        \subsection*{$S_5$ (Reading operands (for ADI))}
        \begin{center}
            \begin{tabular}{|c|c|}
                \hline
                Data Transfer & Commands \\
                \hline
                $T1_{11-9} \to$ RF\_A1  & T2\_E\\
                RF\_D1 $\to$ T2 & T3\_E\\
                $T1_{5-0} \to$ SE\_6 $\to$ T3 & \\
                \hline
            \end{tabular}
        \end{center}
        % states for Atharva's instructions till here

        \subsection*{$S_6$ (Evaluating condition for BEQ)} %BEQ by Harshit % without changing ZF
        \begin{center}
            \begin{tabular}{|c|c|}
                \hline
                Data Transfer & Commands \\
                \hline
                T2 $\to$ ALU\_A & ALU\_J $\leftarrow$ 11\\
                T3 $\to$ ALU\_B & T2\_E\\
                Z\_int $\to$ SE\_1 $\to$ T2 & \\
                ALU\_CND $\leftarrow$ 00 & \\
                \hline
            \end{tabular}
        \end{center}
        \subsection*{$S_7$ (Updating PC in BEQ)} %BEQ by Harshit
        \begin{center}
            \begin{tabular}{|c|c|}
                \hline
                Data Transfer & Commands \\
                \hline
                PC $\to$ ALU\_A & ALU\_J $\leftarrow$ 00\\
                if($T2_0$ == 0) then +1 $\to$ ALU\_B & PC\_E\\
                else $T1 \to$ SE\_10 $\to$ ALU\_B & \\
                ALU\_CND $\leftarrow$ 11 & \\
                ALU\_S $\to$ PC & \\
                \hline
            \end{tabular}
        \end{center}
        \subsection*{$S_8$ (Storing PC into REG\_A)} %JAL by Harshit
        \begin{center}
            \begin{tabular}{|c|c|}
                \hline
                Data Transfer & Commands \\
                \hline
                $T1_{11-9}$ $\to$ RF\_A3 & RF\_WR\\
                PC $\to$ RF\_D3 & \\
                \hline
            \end{tabular}
        \end{center}
        \subsection*{$S_9$ (Branching PC to the address PC + immediate)} %JAL by Harshit
        \begin{center}
            \begin{tabular}{|c|c|}
                \hline
                Data Transfer & Commands \\
                \hline
                PC $\to$ ALU\_A & ALU\_J\\
                $T1_{8-0} \to$ SE\_9 $\to$ ALU\_B & \\
                ALU\_CND $\leftarrow$ 11 & \\
                ALU\_S $\to$ PC & \\
                \hline
            \end{tabular}
        \end{center}
        \subsection*{$S_{10}$ (Branching PC to the address in REG\_B)} %JLR by Harshit
        \begin{center}
            \begin{tabular}{|c|c|}
                \hline
                Data Transfer & Commands \\
                \hline
                $T1_{8-6} \to$ RF\_A1 & PC\_E\\
                RF\_D1 $\to$ PC & \\
                \hline
            \end{tabular}
        \end{center}
        \subsection*{$S_{11}$ (Executing Load Higher Immediate)} %LHI by Harshit
        \begin{center}
            \begin{tabular}{|c|c|}
                \hline
                Data Transfer & Commands \\
                \hline
                $T1_{11-9} \to$ RF\_A3 & RF\_WR\\
                $T1_{11-9} \to$ PZ\_7 $\to$ RF\_D3 & \\   %PZ = pad zeros
                \hline
            \end{tabular}
        \end{center}
        %Harshit's instructions end here

        \subsection*{$S_{12}$ (Computing address of the memory destination)} %LW by Varad
        \begin{center}
            \begin{tabular}{|c|c|}
                \hline
                Data Transfer & Commands \\
                \hline
                T3 $\to$ ALU\_A & ALU\_J $\leftarrow$ 00\\
                $T1_{5-0} \to$ SE\_16 $\to$ ALU\_B & T3\_E\\
                ALU\_S $\to$ T3 & \\
                \hline
            \end{tabular}
        \end{center}
        \subsection*{$S_{13}$ (Writing to the memory)} %SW by Varad
        \begin{center}
            \begin{tabular}{|c|c|}
                \hline
                Data Transfer & Commands \\
                \hline
                T3 $\to$ M\_add & MWR\\
                T2 $\to$ M\_data & \\
                \hline
            \end{tabular}
        \end{center}
        \subsection*{$S_{14}$ (Reading from memory)} %LW by Varad
        \begin{center}
            \begin{tabular}{|c|c|}
                \hline
                Data Transfer & Commands \\
                \hline
                T3 $\to$ M\_add & MDR\\
                M\_data $\to$ T2 & T2\_E\\
                \hline
            \end{tabular}
        \end{center}
        \subsection*{$S_{15}$ (Writing to the register)} %LW by Varad
            \begin{center}
                \begin{tabular}{|c|c|}
                    \hline
                    Data Transfer & Commands \\
                    \hline
                    $T1_{11-9} \to$ RF\_A3 & RF\_WR\\
                    T2 $\to$ RF\_D3 & \\
                    \hline
                \end{tabular}
            \end{center}

            \subsection*{$S_{16}$ (Initial step of SM)} %SM by Varad
            \begin{center}
                \begin{tabular}{|c|c|}
                    \hline
                    Data Transfer & Commands \\
                    \hline
                    (0000000000000000) $\to$ T2 & T2\_E\\
                    $T1_{11-9} \to$ RF\_A2 & T3\_E\\
                    RF\_D2 $\to$ T3 & \\
                    \hline
                \end{tabular}
            \end{center}            
            \subsection*{$S_{17}$ (Looping step 1 of SM)} %SM by Varad
            \begin{center}
                \begin{tabular}{|c|c|}
                    \hline
                    Data Transfer & Commands \\
                    \hline
                    counter := int($T2_{2-0}$) & MWR\\
                    T3 $\to$ ALU\_A & T3\_E\\
                    +1 $\to$ ALU\_B & ALU\_J $\leftarrow$ 00\\
                    if($T1_{counter}$==1) then & \\
                    \{T3 $\to$ M\_add& \\
                    $T2_{2-0}$ $\to$ RF\_A1 & \\
                    RF\_D1 $\to$ M\_data\ & \\
                    ALU\_S $\to$ T3 \} & \\ 
                    \hline
                \end{tabular}
            \end{center} 
            \subsection*{$S_{18}$ (Looping step 2 of SM)} %SM by Varad
            \begin{center}
                \begin{tabular}{|c|c|}
                    \hline
                    Data Transfer & Commands \\
                    \hline
                    T2 $\to$ ALU\_A & ALU\_J $\leftarrow$ 00\\
                    1 bit $\to$ ALU\_B & T2\_E\\
                    ALU\_C $\to$ T2 & \\
                    \hline
                \end{tabular}
            \end{center} 

            \subsection*{$S_{19}$ (Initial step of LM)} %LM by Varad
            \begin{center}
                \begin{tabular}{|c|c|}
                    \hline
                    Data Transfer & Commands \\
                    \hline
                    (0000000000000000) $\to$ T2 & T2\_E\\
                    $T1_{11-9} \to$ RF\_A3 & T3\_E\\
                    RF\_D3 $\to$ T3 & \\     %only diff b/w this and SM's 1st step is D3 vs D2
                    \hline
                \end{tabular}
            \end{center}    

            \subsection*{$S_{20}$ (Looping step 1 of LM)} %LM by Varad --not sure
            \begin{center}
                \begin{tabular}{|c|c|}
                    \hline
                    Data Transfer & Commands \\
                    \hline
                    counter := int($T2_{2-0}$) & MDR\\
                    $T1_{counter}$ $\to$ RF\_WR & T3\_E\\
                    T3 $\to$ M\_add & ALU\_J $\leftarrow$ 00\\
                    M\_data $\to$ RF\_D3 & \\
                    $T2_{2-0}$ $\to$ RF\_A3 & \\
                    T3 $\to$ ALU\_A &\\
                    +1 $\to$ ALU\_B &\\
                    ALU\_CND $\leftarrow$ 00 & \\
                    if($T1_{counter}$==1) then ALU\_S $\to$ T3& \\
                    \hline
                \end{tabular}
            \end{center} 
            \subsection*{$S_{21}$ (Looping step 2 of LM)} %LM by Varad
            \begin{center}
                \begin{tabular}{|c|c|}
                    \hline
                    Data Transfer & Commands \\
                    \hline
                    T2 $\to$ ALU\_A & ALU\_J $\leftarrow$ 00\\
                    1 bit $\to$ ALU\_B & T2\_E\\        %here 1 must be mentioned as a bit ig
                    ALU\_S $\to$ T2 & \\
                    \hline
                \end{tabular}
            \end{center} 
            %Varad's instructions end here
    \section*{Instructions with their State Diagrams and Control Signals}
        \begin{center}
            \begin{tabular}{|c|c|}
                \hline
                Instruction & State flow\\
                \hline
                ADD & $S_0 \to S_1 \to S_2 \to S_3 \to S_4$\\
                ADC & $S_0 \to S_1 \to S_2 \to S_3 \to S_4$\\
                ADZ & $S_0 \to S_1 \to S_2 \to S_3 \to S_4$\\
                ADI & $S_0 \to S_1 \to S_5 \to S_3 \to S_4$\\
                NDU & $S_0 \to S_1 \to S_2 \to S_3 \to S_4$\\
                NDC & $S_0 \to S_1 \to S_2 \to S_3 \to S_4$\\
                NDZ & $S_0 \to S_1 \to S_2 \to S_3 \to S_4$\\
                LHI & $S_0 \to S_1 \to S_{11}$\\
                LW & $S_0 \to S_1 \to S_2 \to S_{12} \to S_{14} \to S_{15}$\\
                SW & $S_0 \to S_1 \to S_2 \to S_{12} \to S_{13}$\\
                & \\
                SM & $S_0 \to S_1 \to S_{16} \to S_{17} $  $ S_{18}$\\
                & \\
                LM & $S_0 \to S_1 \to S_{19} \to S_{20} $  $ S_{21}$\\
                & \\
                BEQ & $S_0 \to S_2 \to S_6 \to S_7$\\
                JAL & $S_0 \to S_8 \to S_9$\\
                JLR & $S_0 \to S_8 \to S_{10}$\\
                \hline
            \end{tabular}
        \end{center} 
        \begin{figure}[H]
            \begin{center}
                \includegraphics*[width=0.8\textwidth]{StateDiagram.jpg}
                \caption{State Transition Diagram}
            \end{center}
        \end{figure}
            



\end{document}
